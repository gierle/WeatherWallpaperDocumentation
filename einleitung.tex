Aufgrund einer globalen Pandemie und weit verbreiteten Schutzmaßnahmen waren viele Arbeitnehmer, Studierende und sogar auch Schüler gezwungen, ihre tagtägliche Arbeit nun von Zuhause aus zu verrichten.
Nach längerer Selbststudie mussten wir leider alle merken, dass Home-Office nicht \enquote{Arbeiten von der eigenen Wohnung aus} sondern eher  \enquote{Wohnen auf der Arbeit} ist.\\
\\
Viele arbeiten deutlich länger, da die geographische Trennung zwischen \textit{Zuhause} und \textit{Arbeit} fehlt.
Vor allem in den Wintermonaten macht es sich bemerkbar, wenn die überhaupt wenigen Sonnenstunden eines Tages von Teams-Meetings durchzogen sind.
Um dieser tristen und grauen Welt etwas Farbe einzuhauchen gibt es nun: \textbf{WeatherWallpaper}\\
\\
Dieses Programm bezieht aktuelle Wetterdaten eines definierten Standortes und zeigt je nach Wetterlage und Uhrzeit ein passendes Bild als Hintergrund an.
Dadurch gleicht sich der Desktop eines jeden Nutzers an die aktuelle natürliche Stimmung, wodurch ein einheitliches Gefühl zwischen Arbeitsplatz und Umwelt vermittelt wird.
Dies wiederum wertet das Klima nicht nur bei der Arbeit, sondern auch Zuhause spürbar auf.
%TODO
%\begin{itemize}
%	\item Code Smells identifizieren
%	\item $>= 2$ Refactoring anwenden und begründen
%\end{itemize}
Bei Refactoring geht es um die Umgestaltung zur Verbesserung der Codequalität, wobei die Gesamtfunktionalität bestehen bleibt.
Im Folgenden werden vier Code Smells und Refactoring Methoden angewandt und beschrieben.


\subsection{Duplicated Code}
Ein bekannter Code Smell ist doppelt vorhandener Code.
Dies hat zur Folge, dass dieser Code auch an beiden Stellen immer gleich gewartet werden muss, also auch doppelten Aufwand fordert.
Vergisst man hierbei eine der Stellen, sind beispielsweise Sicherheitslücken weiterhin vorhanden.
Trotzdem ist dieser Smell sehr einfach zu beseitigen: man muss den Code zusammenführen.\\
\\
Wie im Kapitel \ref{sec:dry} bei dem Prinzip von DRY bereits erklärt, wurden die zwei Implementierungen des Interfaces \texttt{IAPICaller}  zu einer Klasse zusammengefasst.
Dies war ebenfalls eine Code-Duplizierung, welche durch den dort referenzierten Commit gelöst wurde.

\subsection{Large Class}
Dieser Code Smell beschäftigt sich mit der Größe einer Klasse.
Bei einer zu großen Klasse kann es häufig sein, dass das SRP von SOLID (siehe Kapitel \ref{sec:srp}) verletzt oder die Grundkonzepte bei GRASP (siehe \ref{sec:lc_hc}) - Low Coupling und High Cohesion - nicht erfüllt werden.
Daher muss dieser Smell untersucht und bei Problemen die Klasse in Teilfunktionalitäten aufgeteilt werden.\\
Bei der Vorstellung des SRP wurde eine solche Aufteilung vorgenommen und bereits genauer erklärt.
Zusammengefasst wurde die Klasse \texttt{MainController} in vier neue Klassen aufgeteilt, da das SRP nicht mehr erfüllt war.
Hierbei wurde in Refresh-Logik, Timer-Verwaltung, manuelle Aktualisierung mit BackgroundWorker und der GUI-Logik unterschieden.

%\subsection{Extract Method}
% tbd

\subsection{Rename Method}
Dies ist zwar kein eigener \enquote{spezieller} Code Smell, jedoch werden mit dieser Methode unpassende Methodennamen geändert.
Eigentlich liegt hier der Smell \textit{Code Comments} zugrunde, also dass Funktionen einen undeutlichen Namen besitzen und unter anderem deshalb Kommentare im Code nötig sind.\\
In unserem Fall wurde es jedoch angewendet, da zwei eigenständige Methoden unterschiedlicher Klassen zwar in der Praxis ähnliches machen, aber unterschiedlich benannt wurden.
Es handelt sich hierbei um die Klassen \texttt{ImageHandler} und \texttt{WeatherHandler}.
Beide müssen mit ihrer Information jeweils einen Route-String zusammenbauen.
Hierfür existierte beim \texttt{ImageHandler} die Funktion \texttt{GetQueryStringFromAttributes(WeatherInterpretation)} und beim \texttt{WeatherHandler} die Funktion \texttt{BuildRouteString(List<string>)}.
Aus Gründen der Übersichtlichkeit und Leserlichkeit wurde für beide der Name \texttt{BuildRouteString()} normalisiert\footnote{Dieses Refactoring wurde im Commit \url{https://github.com/Bronzila/WeatherWallpaper/commit/8229814773f5c1a59d02521d873cab393e4536bd} durchgeführt.}.

\subsection{Error Code}
Hierbei handelt es sich um schlechtes Code-Design. 
Wird bei einem Fehler spezielle Rückgabewerte geliefert, müssen diese bei jedem Aufruf abgeprüft werden.
Dadurch existiert nicht nur zusätzlicher Code, sondern es wird nicht in Normalfall und Fehlerfall unterschieden.
Hierfür können Exceptions ausgelöst werden.
Diese Methode nennt man \textit{Replace Error Code with Exception}\\
\\
In der Klasse \texttt{ConfigHandler} wurde beim Lesen und Dekodieren einer gespeicherten Konfiguration \texttt{null} zurückgegeben, falls hierbei ein Fehler auftrat. 
Die alte Version des Lesevorgangs wurde im Codebeispiel \ref{lst:ReadConfigFromFile_old} dargestellt.
Falls dies beim Aufruf nicht überprüft und abgefangen wird, können NullPointerExceptions die Ausführung des Programms stoppen.\\
Daher wurden diese Funktionen nach der Methode überarbeitet; die \texttt{BadConfigException()} mit entsprechender Nachricht wird ausgelöst\footnote{Diese Änderungen finden sich im Commit \url{https://github.com/Bronzila/WeatherWallpaper/commit/e3367e008dd79b735d30f64282d8edb929810693} wieder.}.
Dies ist im Listing \ref{lst:ReadConfigFromFile_new} deutlich gemacht.

\begin{listing}[h]
	\inputminted[linenos=true,frame=lines]{csharp}{Listings/ReadConfigFromFile_old.cs}
	\caption{\texttt{ReadConfigFromFile()} mit \texttt{null} als Rückgabewert}
	\label{lst:ReadConfigFromFile_old}
\end{listing}

\begin{listing}[h]
	\inputminted[linenos=true,frame=lines]{csharp}{Listings/ReadConfigFromFile_new.cs}
	\caption{\texttt{ReadConfigFromFile()} überarbeitet nach Replace Error Code with Exception}
	\label{lst:ReadConfigFromFile_new}
\end{listing}

\subsection{Weitere}
Die folgenden Code Smells wurden zwar analysiert, jedoch entweder nicht direkt umgesetzt oder sind nicht existent.
\paragraph{Long Method}
Die Long Method, ähnlich wie \textit{Long Class}, bezieht sich auf die Größe einer Methode.
Hierbei führen lange Methoden zu schlechterer Lesbarkeit und man spart sich Kommentare.
Dies wurde zu teilen bei dem \texttt{WeatherHandler} gefunden, welcher die Methode \texttt{GetWeatherData(Config)} besitzt (siehe Listing \ref{lst:GetWeatherData}).
Hier könnte man die Zeilen 3-5 zusammengefasst werden als eigene Funktion, welche sich um den Zusammenbau des Routestrings kümmert, wie es beim \texttt{ImageHandler} der Fall ist (siehe Listing \ref{lst:ImageHandler_new}).\\
Diese Klassen haben jedoch ein komplett anderes Problem, und zwar \textit{Duplicated Code}. 
Ähnlich wie oben bei diesem Code Smell beschrieben, müssten beide Klassen zusammengefasst werden.
Dies wurde aus zeitlichen Gründen nicht weiter verfolgt.

\begin{listing}[h]
	\inputminted[linenos=true,frame=lines]{csharp}{Listings/GetWeatherData.cs}
	\caption{\texttt{GetWeatherData(Config)}}
	\label{lst:GetWeatherData}
\end{listing}
\paragraph{Shotgun Surgery}
Dieser Code Smell basiert eigentlich auf anderen Smells. 
Um eine Änderung durchführen zu können, muss der Code hier an vielen verschiedenen Stellen im Programm geändert werden.
Dies wurde von uns nicht eindeutig als solches identifiziert, jedoch kann es im Zusammenhang mit dem, in \textit{Long Method} beschriebenen Problem auftreten.
\paragraph{Switch Statement}
\texttt{Switch}-Anweisungen wurden nicht verwendet.
\paragraph{Code Comments}
Erklärende Kommentare wurden nur im äußersten Notfall eingesetzt.
Es ist keine weitere Reduzierung dieser notwendig.